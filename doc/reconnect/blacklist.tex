\documentclass{article}
\usepackage[margin=2.5cm]{geometry}
\usepackage{amsmath, amssymb}
\usepackage{mathpazo, times}

\newcommand{\sLam}[2]{\lambda {#1} \cdot {#2}}
\newcommand{\sApp}[2]{{#1} {#2}}
\newcommand{\sBind}[2]{{#1} \mathrel{\texttt{>}\!\!\texttt{>}\!\texttt{=}} {#2}}
\newcommand{\sReturn}{\mathtt{return}}

\newcommand{\sExpect}{\mathtt{expect}}
\newcommand{\sSend}{\mathtt{send}}
\newcommand{\sSpawn}{\mathtt{spawn}}
\newcommand{\sLink}{\mathtt{link}}
\newcommand{\sThrow}{\mathtt{throw}}
\newcommand{\sReconnect}{\mathtt{reconnect}}

\newcommand{\sParP}{\mathrel{{\parallel}_P}}
\newcommand{\sParN}{\mathrel{{\parallel}_N}}
\newcommand{\sProc}[2]{{#1}_{#2}}
\newcommand{\sPid}{\ensuremath{\mathit{pid}}}
\newcommand{\sNid}{\ensuremath{\mathit{nid}}}
\newcommand{\sLinks}{\ensuremath{\mathit{links}}}
\newcommand{\sNode}[3]{\left[{#1} ; {#2}\right]_{#3}} 
\newcommand{\sSystem}[3]{\left\langle #1 ; #2 ; #3 \right\rangle}
\newcommand{\sNodes}{\mathcal{N}}
\newcommand{\sQueue}{\mathcal{Q}}
\newcommand{\sProcesses}{\mathcal{P}}
\newcommand{\sBlacklist}{\mathcal{B}}

\newcommand{\sJust}[1]{\mathtt{Just} \; {#1}}
\newcommand{\sNothing}{\mathtt{Nothing}}

\newcommand{\sCtxt}[1]{\mathbb{#1}}

\newcommand{\sSenders}{\mathit{senders}}

\newcommand{\OR}{\mathrel{|}}


\begin{document}

\title{Cloud Haskell Semantics}
\author{Edsko de Vries, Well-Typed LLP}
\date{\today}

\maketitle

\section{Syntax}

\subsection{Terms}

\begin{equation*}
\begin{array}{ll@{\;::=\;}l}
\text{value} & V    & \sLam{x}{M} \OR
                      \sExpect \OR
                      \sSend \; \sPid \; V \OR
                      \sSpawn \; \sNid \; V \OR 
                      \sReturn \; M
\\
\text{term}  & M, N & \sApp{M}{N} \OR
                      \sBind{M}{N}
\end{array}
\end{equation*}

\subsection{Processes, Nodes, Systems}

A \emph{process} $\sProc{M}{\sPid}$ is a pair of a term $M$ and a process ID $\sPid$.
A \emph{node}
  $$\sNode{\sProcesses}
          {\sLinks}
          {\sNid}$$
is a tuple consisting of a set of
processes $\sProcesses$, sometimes written as
  $$\sProc{M}{\sPid} \sParP \cdots \sParP \sProc{N}{\sPid'}$$
together with a set \sLinks{} of pairs of process IDs and a node ID \sNid. 
A \emph{system} 
  $\sSystem{\sNodes}{\sQueue}{\sBlacklist}$ 
finally is a tuple containing a set $\sNodes$ of nodes, sometimes written
  $$\sNode{\sProcesses}{\sLinks}{\sNid} 
    \sParN
       \cdots
    \sParN
    \sNode{\sProcesses'}{\sLinks'}{\sNid'}
  $$
a \emph{system queue} $\sQueue$, and a \emph{blacklist} $\sBlacklist$. The
system queue is a set of triples $(\mathit{to}, \mathit{from},
\mathit{message})$ of messages that have been sent but not yet processed. The
blacklist is a set of pairs $(\mathit{to}, \mathit{from})$. Once a pair
$(\mathit{to}, \mathit{from})$ has been blacklisted messages from
$\mathit{from}$ to $\mathit{to}$ will never be delivered ($\mathit{to}$ and
$\mathit{from}$ are node or process identifiers). 

\section{Semantics}

\subsection{Contexts}

We define an evaluation context as usual (TODO: catch?)

\begin{equation*}
\sCtxt{E} ::= [] \OR \sBind{\sCtxt{E}}{M}
\end{equation*}

A process context, parametrized by a \sPid, identifies a process within a set
of nodes. 

\begin{equation*}
\sCtxt{P}_\sPid ::= \sNode{\sProc{\sCtxt{E}}{pid} \sParP \sProcesses}{\sLinks}{\sNid} \sParN \sNodes 
\end{equation*}

We also find it useful to have a process context with two holes, one for a
process identified by \sPid{} and one for the set of processes on node \sNid.
This set of processes may or may not live on the same node as \sPid. 

\begin{equation*}
\begin{array}{rl}
\sCtxt{P}_{\sPid,\sNid} ::= & 
  \sNode{\sProc{\sCtxt{E}}{pid} \sParP []}{\sLinks}{\sNid} \sParN \sNodes \\[1em]
\OR & 
  \sNode{\sProc{\sCtxt{E}}{pid} \sParP \sProcesses}{\sLinks}{\sNid'} \sParN \sNode{[]}{\sLinks}{\sNid} \sParN \sNodes 
    \qquad
  (\sNid \neq \sNid')
\end{array}
\end{equation*}

\subsection{System Rules}

\begin{equation*}
\frac{
}{
  \sSystem{\sNodes}{\sQueue}{\sBlacklist}
\rightarrow
  \sSystem{\sNodes}{\sQueue}{\sBlacklist, (\mathit{to}, \mathit{from})}
} \textsc{Disconnect}
\end{equation*}

\subsection{Process rules}

\subsubsection{Sending}

If the communication is blacklisted the send is a no-op.

\begin{equation*}
\frac{
(\mathit{to}, \mathit{from}) \notin \sBlacklist
}{
  \sSystem{\sCtxt{P}[ \sSend \; \mathit{to} \; V ]_\mathit{from}}
          {\sQueue}
          {\sBlacklist}
\rightarrow 
  \sSystem{\sCtxt{P}[ \sReturn \; () ]_\mathit{from}}
          {\sQueue, (\mathit{to}, \mathit{from}, V)}
          {\sBlacklist}
} \textsc{Send-Ok}
\end{equation*}

\begin{equation*}
\frac{
(\mathit{to}, \mathit{from}) \in \sBlacklist
}{
  \sSystem{\sCtxt{P}[ \sSend \; \mathit{to} \; V ]_\mathit{from}}
          {\sQueue}
          {\sBlacklist}
\rightarrow 
  \sSystem{\sCtxt{P}[ \sReturn \; () ]_\mathit{from}}
          {\sQueue}
          {\sBlacklist}
} \textsc{Send-No}
\end{equation*}

TODO: this allows to send functions

\subsubsection{Receiving}

We return the \emph{first} message of a randomly chosen sender.

\begin{equation*}
\frac{
  \mathit{from} \notin \sSenders(\sQueue)
}{
  \sSystem{\sCtxt{P}[ \sExpect ]_\mathit{from}}
          {\sQueue, (\mathit{to}, \mathit{from}, V), \sQueue'}
          {\sBlacklist}
\rightarrow
  \sSystem{\sCtxt{P}[ \sReturn \; V ]_\mathit{from}}
          {\sQueue, \sQueue'}
          {\sBlacklist}
} \textsc{Expect}
\end{equation*}

\subsubsection{Spawning}

\begin{equation*}
\frac{
  \sPid' \text{ fresh} \qquad
  (\sNid, \sPid) \notin \sBlacklist
}{
  \sSystem{\sCtxt{P}[ \sSpawn \; \sNid \; M ]_\sPid[\sProcesses]_\sNid}
          {\sQueue}
          {\sBlacklist}
\rightarrow          
  \sSystem{\sCtxt{P}[ \sReturn \; (\sJust{\sPid'}) ]_\sPid[\sProcesses \sParP M_{\sPid'}]_\sNid}
          {\sQueue}
          {\sBlacklist}
} \textsc{Spawn-Ok}
\end{equation*}

\begin{equation*}
\frac{
  (\sNid, \sPid) \in \sBlacklist
}{
  \sSystem{\sCtxt{P}[ \sSpawn \; \sNid \; M ]_\sPid}
          {\sQueue}
          {\sBlacklist}
\rightarrow          
  \sSystem{\sCtxt{P}[ \sReturn \; \sNothing ]_\sPid}
          {\sQueue}
          {\sBlacklist}
} \textsc{Spawn-No}
\end{equation*}

\subsubsection{Reconnect}

\begin{equation*}
\frac{
}{
  \sSystem{\sCtxt{P}[ \sReconnect \; \mathit{id} ]_\sPid}
          {\sQueue}
          {\sBlacklist, (\mathit{id}, \sPid), \sBlacklist'}
\rightarrow
  \sSystem{\sCtxt{P}[ \sReturn \; () ]_\sPid}
          {\sQueue}
          {\sBlacklist, \sBlacklist'}
} \textsc{Reconnect-Ok}
\end{equation*}

\begin{equation*}
\frac{
  (\mathit{id}, \sPid) \notin \sBlacklist
}{
  \sSystem{\sCtxt{P}[ \sReconnect \; \mathit{id} ]_\sPid}
          {\sQueue}
          {\sBlacklist}
\rightarrow
  \sSystem{\sCtxt{P}[ \sReturn \; () ]_\sPid}
          {\sQueue}
          {\sBlacklist}
} \textsc{Reconnect-No}
\end{equation*}

\subsubsection{Administration}

TODO: bind, pure evaluation, maybe exception handling

\end{document}
