\documentclass{article}
\usepackage[margin=2.5cm]{geometry}
\usepackage{amsmath, amssymb}
\usepackage{mathpazo, times}

\newcommand{\sLam}[2]{\lambda {#1} \cdot {#2}}
\newcommand{\sApp}[2]{{#1} {#2}}
\newcommand{\sBind}[2]{{#1} \mathrel{\texttt{>}\!\!\texttt{>}\!\texttt{=}} {#2}}
\newcommand{\sReturn}{\mathtt{return}}

\newcommand{\sExpect}{\mathtt{expect}}
\newcommand{\sSend}{\mathtt{send}}
\newcommand{\sSpawn}{\mathtt{spawn}}
\newcommand{\sLink}{\mathtt{link}}
\newcommand{\sThrow}{\mathtt{throw}}
\newcommand{\sReconnect}{\mathtt{reconnect}}
\newcommand{\sNewChan}{\mathtt{newChan}}
\newcommand{\sSendChan}{\mathtt{sendChan}}
\newcommand{\sReceiveChan}{\mathtt{receiveChan}}
\newcommand{\sMonitor}{\mathtt{monitor}}

\DeclareMathOperator{\sNodeOf}{node}
\DeclareMathOperator{\sProcessOf}{process}

\newcommand{\sSpawned}{\mathtt{spawned}}

\newcommand{\sExtend}[1]{\mathrel{\triangleright_{#1}}}

\newcommand{\sPar}{\mathrel{\parallel}}
\newcommand{\sProc}[2]{{#1}_{#2}}

\newcommand{\sNid}{\ensuremath{\mathit{nid}}}
\newcommand{\sPid}{\ensuremath{\mathit{pid}}}
\newcommand{\sCid}{\ensuremath{\mathit{cid}}}

\newcommand{\sId}{\ensuremath{\mathit{id}}}
\newcommand{\sRef}{\ensuremath{\mathit{ref}}}

\newcommand{\sLinks}{\ensuremath{\mathit{links}}}
\newcommand{\sNode}[3]{\left[{#1} ; {#2}\right]_{#3}} 
\newcommand{\sSystem}[4]{\left\langle #1 ; #2 ; #3 ; #4 \right\rangle}
\newcommand{\sNodes}{\mathcal{N}}
\newcommand{\sQueue}{\mathcal{Q}}
\newcommand{\sProcesses}{\mathcal{P}}
\newcommand{\sBlacklist}{\mathcal{B}}
\newcommand{\sMonitors}{\mathcal{M}}

\newcommand{\sJust}[1]{\mathtt{Just} \; {#1}}
\newcommand{\sNothing}{\mathtt{Nothing}}

\newcommand{\sCtxt}[1]{\mathbb{#1}}

\newcommand{\sSenders}{\mathit{senders}}

\newcommand{\OR}{\mathrel{|}}


\begin{document}

\title{Cloud Haskell Semantics (DRAFT)}
\author{Edsko de Vries, Well-Typed LLP}
\date{\today}

\maketitle

\section{Introduction}

This is a semantics for Cloud Haskell, based on the \textit{Unified Semantics
for a Future Erlang} (Hans Svensson, Lars-\AA{}ke Fredlund and Clara Benac
Earle, Erlang '10) but using the semantics style of \textit{Composable Memory
Transactions} (Tim Harris, Simon Marlow, Simon Peyton-Jones, Maurice Herlihy,
PPoPP '05). It differs from the former in that
%
\begin{enumerate}
\item We introduce an explicit notion of \textit{reconnecting} (with potential
message loss)
\item We simplify the semantics: we ``flatten'' sets of nodes of processes as
sets of processes (but assume a mapping from process identifiers to node
identifiers), do not have per-process mailboxes (but only the system queue or
``ether'') and do not have an explicit concept of node controllers
\end{enumerate}
%
It differs from the latter in that
%
\begin{enumerate}
\item We pretend that the Cloud Haskell \texttt{Process} monad is the top-level
monad, and do not consider the \texttt{IO} monad at all
\item We ignore exceptions
\end{enumerate}
%
Current imprecisions with respect to the ``real'' Cloud Haskell are
%
\begin{enumerate}
\item We ignore the issue of serializability, other than to say that the
semantics will get stuck when trying to send a non-serializable payload;
consequently, we do not formalize $\mathtt{static}$
\item We do not formalize all Cloud Haskell primitives (merging of ports,
``advanced messaging'', and others)
\item Some of the concepts that we do formalize are lower-level concepts; for
instance, the primitive $\sSpawn$ that we formalize is asynchronous (following
the Unified semantics); a synchronous construct can be derived.
\end{enumerate}

\section{Preliminaries}

\subsection{Terms}

We assume the usual definitions of values and terms.

\begin{equation*}
\begin{array}{ll@{\;::=\;}l}
\text{value} & V    & \sLam{x}{M} \OR
                      \sExpect \OR
                      \sSend \; \sPid \; M \OR
                      \sSpawn \; \sNid \; M \OR 
                      \sReturn \; M \OR
                      \cdots
\\
\text{term}  & M, N & \sApp{M}{N} \OR
                      \sBind{M}{N}
\end{array}
\end{equation*}

\subsection{Processes, Nodes, Systems}

Processes run on \emph{nodes}. We assume that nodes run a special process
called the \emph{node controller} which is responsible for issuing monitor
messages, spawning new processes, etc. In the semantics messages to node
controllers (for instance, requests to spawn new processes) are recorded as
messages sent to the node itself.

We assume disjoint countable sets $\mathtt{NodeId}$, $\mathtt{ProcessId}$, and
$\mathtt{ChannelId}$, changed over by \sNid, \sPid{} and \sCid{} respectively,
and representing process identifiers, node identifiers, and (typed) channel
identifiers. We assume the existence of total functions
%
\begin{equation*}
\begin{array}{r@{\;:\;}l}
\sNodeOf    & (\mathtt{ProcessId} \uplus \mathtt{ChannelId}) \rightarrow \mathtt{NodeId} \\
\sProcessOf & \mathtt{ChannelId} \rightarrow \mathtt{ProcessId} 
\end{array}
\end{equation*}
%
and define 
$$\mathtt{Identifier} = \mathtt{NodeId} \uplus \mathtt{ProcessId} \uplus \mathtt{ChannelId}$$ 
and let $\sId$ range over $\mathtt{Identifier}$.

We represent a process as a pair $\sProc{M}{\sPid}$ of a term $M$ and a process
ID $\sPid$. We will denote a set of processes as
%
  $$\sProc{M}{\sPid} \sPar \cdots \sPar \sProc{N}{\sPid'}$$
%
A \emph{system} 
  $\sSystem{\sProcesses}{\sQueue}{\sBlacklist}{\sMonitors}$ 
is a tuple containing a set $\sProcesses$ of processes, a \emph{system
queue} $\sQueue$, a \emph{blacklist} $\sBlacklist$, and a set of monitors
$\sMonitors$.
The set of monitors $\sMonitors$ is a set of tuples 
  $(\sId_\mathit{to}, \sPid_\mathit{fr}, \sNid, \sRef)$ 
which records that the node controller at \sNid{} knows that process
$\sPid_\mathit{fr}$ is monitoring $\sId_\mathit{to}$ ($\sRef$ is the monitor
reference).
The system queue is a set of triples $(\sId_\mathit{to}, \sId_\mathit{fr},
\mathit{message})$ of messages that have been sent but not yet processed. The
blacklist records disconnections and is represented as a of pairs
$(\sId_\mathit{to}, \sId_\mathit{fr})$. 

\section{Semantics}

\subsection{Contexts}

We define an evaluation context as in the \textit{Composable Memory
Transactions} paper, but note that we work with \emph{sets} of processes so the
definition of $\sCtxt{P}$ can be slightly simpler. 

\begin{align*}
\sCtxt{E} & ::= [] \OR \sBind{\sCtxt{E}}{M}  \\
\sCtxt{P}_\sPid  & ::= \sCtxt{E}[]_\sPid \sPar \sProcesses
\end{align*}

\subsection{Disconnect}

We assume that entire \emph{nodes} get disconnected from each other, not
individual processes. \emph{Reconnecting} however is on a per process basis. To
formalize this, let $\overline{\sNid}$ be the smallest set, given a set of
processes $\sProcesses$, such that
%
\begin{equation*}
\frac{
}{
\sNid \in \overline{\sNid}
} 
%
\qquad
%
\frac{
\sPid \in \sProcesses \qquad
\sNodeOf(\sPid) = \sNid
}{
\sPid \in \overline{\sNid}
}
%
\qquad
%
\frac{
\sCid \in \sProcesses \qquad
\sNodeOf(\sCid) = \sNid
}{
\sCid \in \overline{\sNid}
}
\end{equation*}

The following rule models random network disconnect between nodes $\sNid_1$ and
$\sNid_2$:

\begin{equation*}
\frac{
}{
  \sSystem{\sProcesses}{\sQueue}{\sBlacklist}{\sMonitors}
\rightarrow
  \sSystem{\sProcesses}{\sQueue}{\sBlacklist \cup (\overline{\sNid_1} \times \overline{\sNid_2})}{\sMonitors}
} \textsc{Disconnect}
\end{equation*}

Once a connection has been blacklisted, no further messages can be sent across
that connection (until an explicit or implicit reconnect). To model this we
introduce a function
%
\begin{equation*}
\begin{array}{l@{\;=\;}l@{\hspace{3em}}l}
  \sQueue \sExtend{\sBlacklist} (\sId_\mathit{to}, \sId_\mathit{fr}, M) 
& 
  \sQueue,  (\sId_\mathit{to}, \sId_\mathit{fr}, M)
&
  \text{if } (\sId_\mathit{to}, \sId_\mathit{fr}) \notin \sBlacklist
\\
  \sQueue \sExtend{\sBlacklist} (\sId_\mathit{to}, \sId_\mathit{fr}, M) 
& 
  \sQueue
&
  \text{otherwise}
\end{array}
\end{equation*}
%
$(\sExtend{\sBlacklist})$ is only defined for serializable payloads.

We provide a primitive to remove a connection from the blacklist:

\begin{equation*}
\frac{
}{
  \sSystem{\sCtxt{P}[ \sReconnect \; \sId_\mathit{to} ]_{\sPid_\mathit{fr}}}
          {\sQueue}
          {\sBlacklist}
          {\sMonitors}
\rightarrow
  \sSystem{\sCtxt{P}[ \sReturn \; () ]_\sPid}
          {\sQueue}
          {\sBlacklist \backslash (\sId_\mathit{to}, \sPid_\mathit{fr})}
          {\sMonitors}
} \textsc{Recon-Ex}
\end{equation*}

Note that using this primitive means giving up on the reliability guarantee: a
call to $\sReconnect$ means that message loss is accepted.

Connections to and from node controllers can be implicitly reconnected:

\begin{equation*}
\frac{
}{
  \sSystem{\sProcesses}{\sQueue}{\sBlacklist, (\sNid_\mathit{to}, \sId_\mathit{fr})}{\sMonitors}
\rightarrow
  \sSystem{\sProcesses}{\sQueue}{\sBlacklist}{\sMonitors}
} \textsc{Recon-Im1}
\qquad
\frac{
}{
  \sSystem{\sProcesses}{\sQueue}{\sBlacklist, (\sId_\mathit{to}, \sNid_\mathit{fr})}{\sMonitors}
\rightarrow
  \sSystem{\sProcesses}{\sQueue}{\sBlacklist}{\sMonitors}
} \textsc{Recon-Im2}
\end{equation*}

\subsection{Send and Receive}

\begin{equation*}
\frac{
}{
  \sSystem{\sCtxt{P}[ \sSend \; \sPid_\mathit{to} \; M ]_{\sPid_\mathit{fr}}}
          {\sQueue}
          {\sBlacklist}
          {\sMonitors}
\rightarrow 
  \sSystem{\sCtxt{P}[ \sReturn \; () ]_{\sPid_\mathit{fr}}}
          {\sQueue \sExtend{\sBlacklist} (\sPid_\mathit{to}, \sPid_\mathit{fr}, M)}
          {\sBlacklist}
          {\sMonitors}
} \textsc{Send}
\end{equation*}

We return the \emph{first} message of a randomly chosen sender.

\begin{equation*}
\frac{
  \sId_{\mathit{fr}} \notin \sSenders(\sQueue)
}{
  \sSystem{\sCtxt{P}[ \sExpect ]_{\sPid_\mathit{to}}}
          {\sQueue, (\sPid_\mathit{to}, \sId_\mathit{fr}, M), \sQueue'}
          {\sBlacklist}
          {\sMonitors}
\rightarrow
  \sSystem{\sCtxt{P}[ \sReturn \; M ]_{\sPid_\mathit{to}}}
          {\sQueue, \sQueue'}
          {\sBlacklist}
          {\sMonitors}
} \textsc{Expect}
\end{equation*}

\subsection{Spawning}

Spawning is asynchronous

\begin{equation*}
\frac{
  \sRef \text{ fresh} 
}{
  \sSystem{\sCtxt{P}[ \sSpawn \; \sNid \; M ]_\sPid}
          {\sQueue}
          {\sBlacklist}
          {\sMonitors}
\rightarrow          
  \sSystem{\sCtxt{P}[ \sReturn \; \sRef ]_\sPid}
          {\sQueue \sExtend{\sBlacklist} (\sNid, \sPid, \sSpawn \; \sRef \; M)}
          {\sBlacklist}
          {\sMonitors}
} \textsc{Spawn-Async}
\end{equation*}

\begin{equation*}
\frac{
  \sPid \notin \sSenders(\sQueue)
\qquad
  \sPid' \text{ fresh} 
\qquad
  \sNodeOf(\sPid') = \sNid
}{
  \sSystem{\sProcesses}
          {\sQueue, (\sNid, \sPid, \sSpawn \; \sRef \; M), \sQueue'}
          {\sBlacklist}
          {\sMonitors}
\rightarrow
  \sSystem{\sProcesses \sPar M_{\sPid'}}
          {\sQueue, \sQueue' \sExtend{\sBlacklist} (\sPid, \sNid, \sSpawned \; \sRef \; \sPid')}
          {\sBlacklist}
          {\sMonitors}
} \textsc{Spawn-Reply}
\end{equation*}

\subsection{Typed Channels}

Since the semantics is not type-driven, we represent typed channels simply as
an identifier with an annotation whether it is the send-end ($\sCid^s$) or the
receive end ($\sCid^r$) of the channel.

\begin{equation*}
\frac{
  \sCid \text{ fresh}
\qquad
  \sProcessOf(\sCid) = \sPid
}{
  \sSystem{\sCtxt{P}[\sNewChan]_\sPid}
          {\sQueue}
          {\sBlacklist}
          {\sMonitors}
\rightarrow
  \sSystem{\sCtxt{P}[\sReturn \; (\sCid^s, \sCid^r)]_\sPid}
          {\sQueue}
          {\sBlacklist}
          {\sMonitors}
} \textsc{NewChan}
\end{equation*}

\begin{equation*}
\frac{
}{
  \sSystem{\sCtxt{P}[ \sSendChan \; \sCid_\mathit{to}^s \; M ]_{\sPid_\mathit{fr}}}
          {\sQueue}
          {\sBlacklist}
          {\sMonitors}
\rightarrow 
  \sSystem{\sCtxt{P}[ \sReturn \; () ]_{\sPid_\mathit{fr}}}
          {\sQueue \sExtend{\sBlacklist} (\sCid_\mathit{to}, \sPid_\mathit{fr}, M)}
          {\sBlacklist}
          {\sMonitors}
} \textsc{SendChan}
\end{equation*}

\begin{equation*}
\frac{
  \sPid_{\mathit{fr}} \notin \sSenders(\sQueue)
}{
  \sSystem{\sCtxt{P}[ \sReceiveChan \; \sCid_\mathit{to}^r ]_{\sPid}}
          {\sQueue, (\sCid_\mathit{to}, \sPid_\mathit{fr}, M), \sQueue'}
          {\sBlacklist}
          {\sMonitors}
\rightarrow
  \sSystem{\sCtxt{P}[ \sReturn \; M ]_{\sPid}}
          {\sQueue, \sQueue'}
          {\sBlacklist}
          {\sMonitors}
} \textsc{ReceiveChan}
\end{equation*}

\subsection{Monitoring}

The local node controller is notified immediately of the new monitor, and the
remote monitor is sent a message (which may, of course, be lost). 

\begin{equation*}
\frac{
  \sNid_\mathit{fr} = \sNodeOf(\sPid_\mathit{fr}) 
\qquad
  \sNid_\mathit{to} = \sNodeOf(\sId_\mathit{to})
\qquad
  \sRef \text{ fresh}
}{
\begin{array}{ll}
& \sSystem{\sCtxt{P}[ \sMonitor \; \sId_\mathit{to} ]_{\sPid_\mathit{fr}}}
          {\sQueue}
          {\sBlacklist}
          {\sMonitors}
\\         
\rightarrow &
  \sSystem{\sCtxt{P}[ \sReturn \; \sRef ]_{\sPid_\mathit{fr}}}
          {\sQueue \sExtend{\sBlacklist} (\sNid_\mathit{to}, \sPid_\mathit{fr}, \sMonitor \; \sRef \; \sId_\mathit{to})}
          {\sBlacklist}
          {\sMonitors, (\sId_\mathit{to}, \sPid_\mathit{fr}, \sNid_\mathit{fr}, \sRef)}
\end{array}
} \textsc{Monitor}
\end{equation*}

\begin{equation*}
\frac{
  \sPid_\mathit{fr} \notin \sSenders(\sQueue)
}{
  \sSystem{\sProcesses}
          {\sQueue, (\sNid_\mathit{to}, \sPid_\mathit{fr}, \sMonitor \; \sRef \; \sId_\mathit{to}), \sQueue'}
          {\sBlacklist}
          {\sMonitors}
\rightarrow          
  \sSystem{\sProcesses}
          {\sQueue, \sQueue'}
          {\sBlacklist}
          {\sMonitors, (\sId_\mathit{to}, \sPid_\mathit{fr}, \sNid_\mathit{to}, \sRef)}
} \textsc{Monitor-Remote}
\end{equation*}

The \emph{local} node controller notifies a monitoring process when it gets
disconncted from a monitored remote process.

\begin{equation*}
\frac{
  \sNid_\mathit{fr} = \sNodeOf(\sPid_\mathit{fr})
\qquad
  (\sId_\mathit{to}, \sPid_\mathit{fr}) \in \sBlacklist
}{
  \sSystem{\sProcesses}
          {\sQueue}
          {\sBlacklist}
          {\sMonitors, (\sId_\mathit{to}, \sPid_\mathit{fr}, \sNid_\mathit{fr}, \sRef)}
\rightarrow          
  \sSystem{\sProcesses}
          {\sQueue, (\sPid_\mathit{fr}, \sId_\mathit{to}, \mathtt{discon} \; \sRef)}
          {\sBlacklist}
          {\sMonitors}
} \textsc{Discon}
\end{equation*}

\subsection{Process Termination}

We remove a terminated process only once all its monitors have been notified.

\begin{equation*}
\frac{
  \sNodeOf(\sPid) = \sNid
\qquad
  \nexists \; \sPid', \sRef \cdot (\sPid', \sPid, \sNid, \sRef) \in \sMonitors 
}{
  \sSystem{{(\sReturn \; ())}_\sPid \sPar \sProcesses}
          {\sQueue}
          {\sBlacklist}
          {\sMonitors}
\rightarrow          
  \sSystem{\sProcesses}
          {\sQueue}
          {\sBlacklist}
          {\sMonitors}
} \textsc{Term}
\end{equation*}

The \emph{remote} node controller notifies a monitoring process when its
monitored remote process terminates.

\begin{equation*}
\frac{
  \sNodeOf(\sPid) = \sNid
}{
  \sSystem{{(\sReturn \; ())}_\sPid \sPar \sProcesses}
          {\sQueue}
          {\sBlacklist}
          {\sMonitors, (\sPid, \sPid', \sNid, \sRef)}
\rightarrow
  \sSystem{{(\sReturn \; ())}_\sPid \sPar \sProcesses}
          {\sQueue \sExtend{\sBlacklist} (\sPid', \sPid, \mathtt{died} \; \sRef)}
          {\sBlacklist}
          {\sMonitors}
} \textsc{Died}
\end{equation*}

\subsection{Administration}

TODO: bind, pure evaluation, maybe exception handling

\end{document}
